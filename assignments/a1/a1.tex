\documentclass[10pt,hidelinks]{article}
\usepackage[letterpaper, hmargin=0.75in, vmargin=0.75in]{geometry}
\usepackage{graphicx}
\usepackage[hyphens]{url}
\usepackage{hyperref}
\usepackage{listings}
\usepackage{pgf}
\usepackage{courier}

\parindent 0in
\parskip 1.5ex


\lstset{ %
language=Java,
basicstyle=\ttfamily\scriptsize,commentstyle=\scriptsize\itshape,showstringspaces=false,breaklines=true}


\begin{document}

\title{
SE465 \\
Software Testing, Quality Assurance, and Maintenance\\
Assignment/Lab 1, version 0}
\author{Patrick Lam \\
{Release Date:  January 16, 2019} \\
}
\renewcommand{\today}{}
\maketitle

\begin{center}

{\bf Due:  11:59 PM, Wednesday, January 30, 2019} \\
{\bf Submit: via git.uwaterloo.ca }\\
\end{center}

\section*{Getting set up}
We will create a copy of the starter repo for you in your {\tt git.uwaterloo.ca} account. You need to log in to {\tt git.uwaterloo.ca} for that to work.

You will also want to fork a copy (``remix'') of the Glitch repo of the webapp at
\url{https://glitch.com/~se465-flashcards}. The ``Remix This'' button will let you do that.

An account on {\tt ecelinux.uwaterloo.ca} is available to you.
Several of the resources required for this assignment are already installed on these servers. Probably the Vagrant image is easiest to work with.
If you are attempting to connect to a server from off campus, remember you will need to connect to the University's VPN first: \url{https://uwaterloo.ca/information-systems-technology/services/virtual-private-network-vpn/about-virtual-private-network-vpn}

I expect each of you to do the assignment independently. I will follow UW's Policy 71 for all cases of plagiarism.
 
\newpage
 \section*{Submission instructions:} 
Commit {\bf and push} your modifications back to your fork on {\tt
  git.uwaterloo.ca}.  It's git, so you can submit multiple times. After
submission, {\bf please re-clone your submissions to make sure you
  have uploaded all necessary files}.
 
\section*{Submission summary}
Here's what you need to submit in your fork of the repo. Be sure to commit
and {\bf push} your changes back to {\tt git.uwaterloo.ca}.
\begin{enumerate}
\item your modified {\tt FormattedCommandAliasTest.java} file in path
\url{shared/bukkit/src/test/java/org/bukkit/command}.
\item in directory {\tt q2}, either file {\tt exploratory.pdf} or {\tt exploratory.txt}, respectively
in PDF or text format. I've included {\tt exploratory.tex} which you can
\LaTeX into {\tt exploratory.pdf}. (No Microsoft Word files, please).
\item in directory {\tt q3}, TBA.
\end{enumerate}
 
 \begin{center}
 \begin{tabular}{c|cc}
 Question   &  TA in Charge \\ \hline
1 & (mostly machine, supervised by Meet) \\ 
2 & Jason \\ 
3 & Meet \\ 
4 & Michael \\
5 & Parsa
 \end{tabular}
 \end{center}

\newpage

\section*{Question 1 (10 points)}
For this question, you will write JUnit tests for the {\tt
  FormattedCommandAlias} class of the Bukkit Minecraft server API to
achieve 100\% statement coverage.

The provided Vagrant image includes a slightly modified version of
Bukkit in the \url{~/shared/bukkit} directory (bukkit diff also available in course
github at \url{assignments/a1/bukkit-test-instrumentation.diff}.  
I added a skeleton test class for {\tt
  FormattedCommandAlias}, called {\tt FormattedCommandAliasTest}.

You can run the tests in the VM with the command {\tt mvn test}.  To
generate the Jacoco coverage report, use {\tt mvn package}. You'll
find the resulting reports in
\url{~/shared/bukkit/target/site/jacoco/org.bukkit.command}.

\paragraph{Your task.} Add JUnit unit tests to {\tt FormattedCommandAliasTest}
that achieve 100\% statement 
coverage for the {\tt org.bukkit.command.FormattedCommandAlias} class
and that verify the results of the computation.

Marking scheme: We will mark your modified {\tt
  org.bukkit.command.TestFormattedCommandAlias} class.  5 points for
coverage (full marks for 100\% statement coverage, nonlinearly scaled
down for less), 5 points for your tests having passing assertions that
verify the output. You will get 0 points if your code doesn't compile.

\section*{Questions 2--5: ``se465-flashcards''}
The next 4 questions are about the ``se465-flashcards'' webapp.
I hope that using Glitch will ensure painless setup; when you click the
``remix'' button on \url{https://glitch.com/~se465-flashcards},
it will create a copy of the code and set up a virtual machine for you.

This webapp, written in Express + pug, includes both frontend code (in
the {\tt public/js} and {\tt views} directories), as well as backend code
(in the node.js files in the main directory). 

\section*{Question 2 (10 points)}
In this question, you will perform exploratory testing on ``se465-flashcards''.
The charter will be ``Explore the overall functionality of
se465-flashcards''. (1 point) Summarize in one or two sentences what you
perceive as the goal of ``se465-flashcards''. (5~points) Identify the tasks
that ``se465-flashcards'' should be able to do and classify them as primary or
contributing. (You probably want to do this in parallel with your
exploratory testing). (1 points) Identify areas of potential
instability. (3 points) Produce exploratory testing notes summarizing
your findings (one or two paragraphs); in Question 3 you will report a
bug, so no need to do that here.

\section*{Question 3 (10 points)}
(3 points) Identify a failure (bug) in the ``se465-flashcards'' webapp.
Limit yourself to using the front-end as presented by the application; that is,
don't enter custom URLs or send requests to the back-end.
(2 points) Write down a sequence of
steps to reproduce the bug. (5 points) Implement a fix for the bug and explain the
fix; show the incorrect and correct outputs. (Screenshots are probably
your best bet).

(Think about the program's input space and poke at corners of the
input space.)

\section*{Question 4 (15 points)}
Now we'll talk about the back-end. We talked about ``controllability''
and ``observability'' for systems.  It turns out that the backend API
has endpoints that provide controllability but not observability; the
frontend directly queries the database to get its results (a poor
design!)  (1 point) Briefly summarize what each of the commands in
{\tt apiRouter.js} can do. (4 points) Propose changes to this API that
would make each of the endpoints {\tt /api/update}, {\tt /api/delete},
and {\tt /api/upload} testable. (3 points) Implement these changes and
submit a {\tt diff}.
(7 points) Write automatic test cases for each of the 3 endpoints.
These test cases should ensure that the system is in a known state
before they start running. (Your test cases should target your own instance
of the webapp.)

REST-assured looks like a good way to use JUnit tests for REST
testing: \url{https://github.com/rest-assured/rest-assured}.

\section*{Question 5 (15 points)}
We'll consider the front-end next. (1 points) What feature of the front-end
makes it difficult to test? (2 points) How can we add more controllability
to the front-end to make it easier to test? (3 points) Implement your proposed
change. (9 points) Create a Selenium test suite that exercises 3 features of the
{\tt se465-testcases} webapp.

In the {\tt shared/selenium} directory in your repo, you will find a {\tt SeleniumExample}.
You can run tests from this example using the command:

\begin{verbatim}
mvn test "-Dtest=se465.SeleniumExample#test*" -Dwebdriver.base.url=http://www.google.com
\end{verbatim}

Your test suite should be called {\tt se465.FlashcardsTestSuite} and we should be able to run your tests with this command:

\begin{verbatim}
mvn test "-Dtest=se465.FlashcardsTestSuite#test*"
\end{verbatim}

{\bf HINT:} You can't get the text from an input element with {\tt getText()}. See instead
\url{http://www.w3schools.com/jsref/prop_text_value.asp}.

Useful reference:

\url{https://seleniumhq.github.io/selenium/docs/api/java/org/openqa/selenium/WebElement.html}


\end{document}
