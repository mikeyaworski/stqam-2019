\documentclass[11pt]{article}
\usepackage{listings}
\usepackage{tikz}
\usepackage{alltt}
\usepackage{hyperref}
\usepackage{url}
%\usepackage{algorithm2e}
\usetikzlibrary{arrows,automata,shapes}
\tikzstyle{block} = [rectangle, draw, fill=blue!20, 
    text width=5em, text centered, rounded corners, minimum height=2em]
\tikzstyle{bt} = [rectangle, draw, fill=blue!20, 
    text width=1em, text centered, rounded corners, minimum height=2em]

\newtheorem{defn}{Definition}
\newtheorem{crit}{Criterion}
\newcommand{\true}{\mbox{\sf true}}
\newcommand{\false}{\mbox{\sf false}}

\newcommand{\handout}[5]{
  \noindent
  \begin{center}
  \framebox{
    \vbox{
      \hbox to 5.78in { {\bf Software Testing, Quality Assurance and Maintenance } \hfill #2 }
      \vspace{4mm}
      \hbox to 5.78in { {\Large \hfill #5  \hfill} }
      \vspace{2mm}
      \hbox to 5.78in { {\em #3 \hfill #4} }
    }
  }
  \end{center}
  \vspace*{4mm}
}

\newcommand{\lecture}[4]{\handout{#1}{#2}{#3}{#4}{Lecture #1}}
\topmargin 0pt
\advance \topmargin by -\headheight
\advance \topmargin by -\headsep
\textheight 8.9in
\oddsidemargin 0pt
\evensidemargin \oddsidemargin
\marginparwidth 0.5in
\textwidth 6.5in

\parindent 0in
\parskip 1.5ex
%\renewcommand{\baselinestretch}{1.25}

\usepackage{enumitem}

\lstset{ %
language=Java,
basicstyle=\ttfamily,commentstyle=\scriptsize\itshape,showstringspaces=false,breaklines=true,numbers=left}

\usepackage{fontspec}
\setmonofont{Cousine}[Scale=MatchLowercase]

\begin{document}

\lecture{9 --- January 23, 2019}{Winter 2019}{Patrick Lam}{version 1}

\section*{Selenium\footnote{\url{http://seleniumhq.org}}}
Selenium is a tool for (among other things) testing web applications.
In its full generality, ``Selenium automates browsers'' and lets you
programmatically drive Web browsers.  For the purposes of this class,
you can use Selenium to automate web application tests, testing
multiple browsers and multiple platforms.

In Assignment 1, I asked you to use Selenium to test a web application.
I've simplified the setup by not actually launching a browser; Selenium
contains a ``headless'' (GUI-less) browser simulator called {\tt HtmlUnit}.
Selenium can also drive real browsers, and it's interesting to watch it do that.
Furthermore, you can set up a bunch of different computers and drive them from
a single Selenium script.

Selenium also includes record-replay functionality (the Selenium IDE, a Firefox
add-on), but I won't discuss it.

\paragraph{Anatomy of a Selenium Test.} There are many bindings for Selenium, but
we'll talk about using Selenium from JUnit.

Like any other test, there is setup and teardown; for Selenium, you
ask for the browser to be started and terminated.

\begin{lstlisting}
  @Before
  public void openBrowser() {
    baseUrl = System.getProperty("webdriver.base.url");
    driver = new FirefoxDriver(); // could also be Chrome, IE, HtmlUnit, etc.
    driver.get(baseUrl);
  }

  @After
  public void saveScreenshotAndCloseBrowser() throws IOException {
    driver.quit();
  }
\end{lstlisting}

Note that we create a browser and ask it to load a page.

\newpage
Now let's look at a particular test. Like any other test, it sets up the test state,
calls upon the system under test to do the action, and checks the result.

\begin{lstlisting}
  @Test
  public void testPageTitleAfterSearchShouldBeginWithDrupal() throws IOException {
    assertEquals("The page title should equal Google at the start of the test.",
                 "Google", driver.getTitle());  // (1)
    WebElement searchField = driver.findElement(By.name("q")); // (2)
    searchField.sendKeys("Drupal!"); // (3)
    searchField.submit();            // (3)
    assertTrue("The page title should start with the search string after the search.", // (4)
               (new WebDriverWait(driver, 10)).until(new ExpectedCondition<Boolean>() {
                       public Boolean apply(Object d) {
                           return ((WebDriver)d).getTitle().toLowerCase().startsWith("drupal!");
                       }
                   })
               );
  }
\end{lstlisting}

Here's what's going on.
\begin{enumerate}[noitemsep]
\item The initial {\tt assertEquals} is based on an assumption that the test is called on
  {\tt www.google.com}.
\item Next, the test searches for the appropriate input on the
  webpage. (For Java tests, we don't need to search for the method
  that we're calling. Here, we do.) Selenium includes various ways to
  search the Document Object Model.
\item Having found the appropriate field, the test sends input and submits the form.
  This is analogous to calling the methods on the System Under Test.
\item Finally, the test waits for the action to complete (with {\tt
  WebDriverWait}). Once complete, it checks that the outputs are
  correct; in this case, it checks that the title of the resulting
  page starts with ``drupal!''.
\end{enumerate}

\paragraph{Waiting.} In the above example, the test simply waits until the page loads (or 10 seconds). Selenium tests can also wait for more sophisticated conditions, for instance until a certain element appears on the page. (This is relevant when the JavaScript is adding content to the page dynamically).

Example:
\begin{lstlisting}
WebDriverWait wait = new WebDriverWait(driver, 10);
WebElement element =
  wait.until(ExpectedConditions.elementToBeClickable(By.id("someid")));
\end{lstlisting}

\section*{Page Objects}
The above example hard-coded the ``q'' field on Google's page.
But things change. We can fix that by introducing a level of
indirection.

References:

\url{http://www.seleniumhq.org/docs/06_test_design_considerations.jsp#chapter06-reference}\\
\url{https://martinfowler.com/bliki/PageObject.html}

A page object should abstractly represent the actions that
a user can take on a page and allow the caller to query
the state of the page (if necessary). The Selenium documentation
suggests a page object for a sign-in page:

\begin{lstlisting}
public class SignInPage {
  private WebDriver driver;

  public SignInPage(WebDriver driver) {
    this.driver = driver;
    if(!driver.getTitle().equals("Sign in page")) {
      throw new IllegalStateException("This is not sign in page, current page is: "
      +driver.getLocation());
    }
  }

  public HomePage loginValidUser(String userName, String password) {
    driver.type("usernamefield", userName);
    driver.type("passwordfield", password);
    driver.click("sign-in");
    driver.waitForPageToLoad("waitPeriod");

    return new HomePage(driver);
  }
}
\end{lstlisting}

The documentation suggests that the only verification that
should occur on a page object is testing that the driver
points to the right page. Everything else should be done in
the tests.

The sign-in page object also exposes the sole functionality of the
page, which is to sign in. It then returns the page that gets loaded
after sign-in.

Note that the caller does not need to know about the identity of the
username and password fields. In fact, let's say that a new version of
the app included Facebook OAuth login. It would suffice to change the
{\tt SignInPage} to use the new login functionality; no other parts of the test suite would need to change.

In assignment 2, I will ask you to implement a fairly low-level Page Object, which simply passes the field contents directly to the caller. The sign-in example demonstrates a higher level of abstraction.

\end{document}
