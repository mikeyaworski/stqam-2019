\documentclass[11pt]{article}
\usepackage{url}
\usepackage{listings}
\usepackage{tikz}
\usepackage{fontspec}
\usepackage{enumitem}
\setmainfont{Latin Modern Roman}
\setmonofont{Cousine}[Scale=MatchLowercase]
\usetikzlibrary{arrows,automata,shapes}
\tikzstyle{block} = [rectangle, draw, fill=blue!20, 
    text width=5em, text centered, rounded corners, minimum height=2em]
\tikzstyle{bt} = [rectangle, draw, fill=blue!20, 
    text width=1em, text centered, rounded corners, minimum height=2em]

\newtheorem{defn}{Definition}
\newtheorem{crit}{Criterion}
\newcommand{\true}{\mbox{\sf true}}
\newcommand{\false}{\mbox{\sf false}}

\newcommand{\handout}[5]{
  \noindent
  \begin{center}
  \framebox{
    \vbox{
      \hbox to 5.78in { {\bf Software Testing, Quality Assurance and Maintenance } \hfill #2 }
      \vspace{4mm}
      \hbox to 5.78in { {\Large \hfill #5  \hfill} }
      \vspace{2mm}
      \hbox to 5.78in { {\em #3 \hfill #4} }
    }
  }
  \end{center}
  \vspace*{4mm}
}

\newcommand{\lecture}[4]{\handout{#1}{#2}{#3}{#4}{Lecture #1}}
\topmargin 0pt
\advance \topmargin by -\headheight
\advance \topmargin by -\headsep
\textheight 8.9in
\oddsidemargin 0pt
\evensidemargin \oddsidemargin
\marginparwidth 0.5in
\textwidth 6.5in

\parindent 0in
\parskip 1.5ex
%\renewcommand{\baselinestretch}{1.25}

\usepackage[listings]{tcolorbox}
\newtcbinputlisting{\codelisting}[3][]{
    extrude left by=1em,
    extrude right by=2em,
    listing file={#3},
    fonttitle=\bfseries,
    listing options={basicstyle=\ttfamily\footnotesize,numbers=left,language=Java,#1},
    listing only,
    hbox,
}
\lstset{ %
language=Java,
basicstyle=\ttfamily,commentstyle=\scriptsize\itshape,showstringspaces=false,breaklines=true,numbers=left}


\begin{document}

\lecture{27 --- March 25, 2019}{Winter 2019}{Patrick Lam}{version 1.1}

\section*{Reporting Bugs}

Goal of reporting bugs:

\begin{quote}
  Get bugs fixed. Which bugs? The most important ones---the right ones.
\end{quote}

How?\footnote{\url{http://blog.cleverelephant.ca/2010/03/how-to-get-your-bug-fixed.html}}\footnote{\url{http://blog.threepress.org/2009/11/17/how-to-get-your-bug-fixed/}}\\[7em]

% sell the bug: 
%  * developers don't necessarily have to listen to bug reporters
%  * make life easy for them or catch their interest
%     - submit a patch
%     - post the bug in the right place
%     - make bug easily reproducible
%     - maximize its generality/simplify report as much as possible
%     - be responsibe to further queries
%  * overcome their objections to fixing the bug

% provide a clear report, be polite (use good tone), provide a test case.

\subsection*{Properties of Important Bugs}

\begin{itemize}[noitemsep]
\item Bug is sufficiently general to affect many users (easily reproducible).
\item Bug has severe consequences (crashes, dataloss).
\item Bug is new to most recent version.
\item Bug has security implications.
\end{itemize}

\subsection*{Reproducibility}

Developers can't fix problems they can't observe. 
\begin{itemize}[noitemsep]
\item Be maximally
specific in describing steps to reproduce.
\item Write the steps down as
soon as possible, before you forget. 
\item Try to find a minimal
testcase that demonstrates the problem.
\end{itemize}

\clearpage
\subsection*{Anatomy of a Bug Report}
Bug report formats are fairly standard. Here are some fields
in a Bugzilla report.

Standard bookkeeping fields:
\begin{itemize}[noitemsep]
\item Bug id: automatically generated number
\item Reporter: usually an email address
\item Product, Version, Component: helps direct the bug to the right developers
\item Platform, OS: e.g. PC/Linux, or Mac/Mac OS X 10.5.
\item Severity: based on consequences; examples: blocker, critical, normal, trivial, enhancement
\item Assign to: person responsible for bug
\item CC: list of people who track bug changes
\item Keywords: e.g. crash, intl, patch, security
\item Depends on/blocks: relationships between bugs
\item URL: (especially applicable for browsers)
\item Attachments: e.g. test cases/input files which exhibit the bug
\end{itemize}

\paragraph{Summary.} Perhaps the most critical field: a one-line recap of the bug. Enables searching for and judgement of the bug.

Examples (some good, some bad):
\begin{itemize}[noitemsep]
\item ``RPM 4 installer crashes if launched on Red Hat 6.2 (RPM 3) system''
\item ``Back button does not work''
\item ``When memory cache disabled, no-store pages not displayed at all''
\item ``History and bookmarks completely inoperable''
\item ``Can't install''
\end{itemize}

\paragraph{Description.} Should be a complete description of the bug, 
including:

\begin{itemize}[noitemsep]
\item Overview: expanded summary, e.g. ``Drag-selecting any page crashes 
Mac builds in NSGetFactory''.
\item Steps to Reproduce (key!): minimized easy-to-follow steps to trigger
the bug; 1) try to reproduce the bug based on the steps you report; 2)
try to describe the steps to that anyone (like the developer) can reproduce
the bug. 

Be specific: instead of ``save the document'', 
``File \textgreater~ Save, select foo from dropdown, \ldots''.
\item Actual Results: what you see when you perform the steps to reproduce.
e.g. ``Application crashes. Stack trace included.''
\item Expected Results: what you think is correct
\item Build Date and Platform: on development software, helps find the bug;
include additional builds and platforms the bug might apply to.
\end{itemize}

\paragraph{Lifecycle-related fields.} Some fields summarize the
current state of the bug.
\begin{itemize}[noitemsep]
\item Comments: either by the assigned developer, original reporter, or
interested bystanders. Often people give additional information (``also happens
on latest build'') or help diagnose the problem.
\item Status: e.g. UNCONFIRMED, NEW, REOPENED, VERIFIED
\item Priority: for internal use by development team.
\end{itemize}

\subsection*{Properties of Good Bug Reports}
\begin{itemize}[noitemsep]
\item Reported in the database.
\item Simple: one bug per report.
\item Understandable, minimal, and generalizable.
\item Reproducible.
\item Non-judgemental. ``The developers are all morons''.
\item Not a duplicate.
\end{itemize}

\subsection*{Bug Triage}
Some of the fields we've seen help in identifying the most
important bugs (which ones?). Consider this approach for 
assigning a single number to a bug which summarizes its
importants (``user pain''):

\begin{center}
\url{http://lostgarden.com/2008/05/improving-bug-triage-with-user-pain.html}
\end{center}

The main attributes are type, likelihood, and priority, all evaluated on anchored scales on which staff are calibrated.

\subsection*{Research: What Makes a Good Bug Report?}
Betternburg et al performed a survey asking experienced developers to
rate bug reports and to identify important information in
them~\cite{bettenburg-fse-2008}, published in Foundations of Software
Engineering 2008.

I recommend consulting the full paper. But the executive summary is
that developers rated steps to reproduce (83\%), stack traces (57\%)
and test cases (51\%) as most useful. Furthermore, by examining data
from Apache projects, Eclipse, and Mozilla, they found that bug
reports with stack traces get fixed sooner, and that bug reports that
are easier to read have lower lifetimes. Code samples also help increase
the chance that a bug report gets fixed.

%% \subsection*{What to Report?}
%% Which bugs to report and fix (p127)

%% \subsection*{Biases}
%% Things that bias bug reports (BugAdvocacy.pdf, p132).

\nocite{kaner:_bug_advoc}
\bibliographystyle{alpha}
\bibliography{L27}

\end{document}
