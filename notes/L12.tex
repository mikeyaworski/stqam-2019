\documentclass[11pt]{article}
\usepackage{listings}
\usepackage{tikz}
\usepackage{enumerate}
\usetikzlibrary{arrows,automata,shapes}

\newtheorem{defn}{Definition}
\newtheorem{crit}{Criterion}

\newcommand{\handout}[5]{
  \noindent
  \begin{center}
  \framebox{
    \vbox{
      \hbox to 5.78in { {\bf Software Testing, Quality Assurance and Maintenance } \hfill #2 }
      \vspace{4mm}
      \hbox to 5.78in { {\Large \hfill #5  \hfill} }
      \vspace{2mm}
      \hbox to 5.78in { {\em #3 \hfill #4} }
    }
  }
  \end{center}
  \vspace*{4mm}
}

\newcommand{\lecture}[4]{\handout{#1}{#2}{#3}{#4}{Lecture #1}}
% 1-inch margins, from fullpage.sty by H.Partl, Version 2, Dec. 15, 1988.
\topmargin 0pt
\advance \topmargin by -\headheight
\advance \topmargin by -\headsep
\textheight 8.9in
\oddsidemargin 0pt
\evensidemargin \oddsidemargin
\marginparwidth 0.5in
\textwidth 6.5in

\parindent 0in
\parskip 1.5ex
%\renewcommand{\baselinestretch}{1.25}

\begin{document}

\lecture{12 --- February 4, 2019}{Winter 2019}{Patrick Lam}{version 1}

\section*{Syntax-Based Testing}
We are going to completely switch gears now. We will see two applications of context-free grammars:
\begin{enumerate}
\item input-space grammars: create inputs (both valid and invalid)
\item program-based grammars: modify programs (mutation testing)
\end{enumerate}


\paragraph{Mutation testing.} The basic idea behind mutation testing is to improve your test suites
by creating modified programs (\emph{mutants}) which force your test
suites to include test cases which verify certain specific behaviours
of the original programs by killing the mutants. We'll talk about this
in more detail in a week or two.

\subsection*{Generating Inputs: Regular Expressions and Grammars}

Consider the following Perl regular expression for Visa numbers:

\begin{verbatim}
                        ^4[0-9]{12}(?:[0-9]{3})?$
\end{verbatim}
%(This regexp is different from the syntax you saw in automata/compilers
%classes, but could be expressed in the standard way too, as it has no
%backrefs.)

Idea: generate ``valid'' tests from regexps and invalid tests by 
mutating the grammar/regexp. {\sf (Why did I put valid in quotes? What
is the fundamental limitation of regexps?)}
\\[4em]

%Fundamental limitation of regexps: \\[2em]
% can't count, e.g. can't balance parens.

Instead, we can use grammars to generate inputs (including sequences
of input events).

Typical grammar fragment:

\begin{eqnarray*}
\mbox{mult\_exp} &=& \mbox{unary\_exp} \mid \mbox{mult\_exp {\tt STAR} unary\_arith\_exp} \mid \mbox{mult\_exp {\tt DIV} unary\_arith\_exp}; \\
\mbox{unary\_exp} &=& \mbox{quant\_exp} \mid \mbox{unary\_exp {\tt DOT INT}}
\mid \mbox{unary\_exp up};\\
&\vdots& \\
\mbox{start} &=& \mbox{header}^? \mbox{declaration}^*
\end{eqnarray*}

\paragraph{Using Grammars.} Two ways you can use input grammars for
software testing and maintenance:
\begin{itemize}
\item recognizer: can include them in a program to validate inputs;
\item generator: can create program inputs for testing.
\end{itemize}
Generators start with the start production and replace nonterminals
with their right-hand sides to get (eventually) strings belonging to 
the input languages. Typically you would generate inputs until you have
enough, either randomly sampling from the input space or systematically
generating inputs of larger and larger sizes.

\paragraph{Another Grammar.}~\\

{\sf 
\begin{tabular}{lcl}
roll &=& action$^*$ \\
action &=& dep $\mid$ deb \\
dep &=& {\tt "deposit"} account amount \\
deb &=& {\tt "debit"} account amount \\
account &=& digit \{ 3 \} \\
amount &=& {\tt "\$"} digit$^+$ {\tt "."} digit \{ 2 \} \\
digit &=& ["0" -- "9"]
\end{tabular}
}

\paragraph{Examples of valid strings.}~\\[4em]
%\begin{verbatim} 
% deposit 739 $12.35
% deposit 929 $33.20
% debit 739 $6.00
%\end{verbatim}

Note: creating a grammar for a system that doesn't have one, but should,
is a useful QA exercise. Using this grammar at runtime to validate inputs
can improve software reliability, although it makes tests generated
from the grammar less useful.


\paragraph{Some Grammar Mutation Operators.} 
\begin{itemize}
\item Nonterminal Replacement; e.g. \\
{\sf dep = {\tt "deposit"} account amount $\Longrightarrow$
 dep = {\tt "deposit"} amount amount}\\[4em]
(Use your judgement to replace nonterminals with similar nonterminals.)

\item Terminal Replacement; e.g. \\
{\sf amount = {\tt "\$"} digit$^+$ {\tt "."} digit \{ 2 \} 
$\Longrightarrow$ amount = {\tt "\$"} digit$^+$ {\tt "\$"} digit \{ 2 \} } \\[4em]

\item Terminal and Nonterminal Deletion; e.g.\\
{\sf dep = {\tt "deposit"} account amount $\Longrightarrow$
 dep = {\tt "deposit"} amount}\\[4em]

\item Terminal and Nonterminal Duplication; e.g.\\
{\sf dep = {\tt "deposit"} account amount $\Longrightarrow$
 dep = {\tt "deposit"} account account amount}\\[4em]
\end{itemize}

\paragraph{Using grammar mutation operators.}
\begin{enumerate}
\item mutate grammar, generate (invalid) inputs; or,
\item use correct grammar, but mis-derive a rule once---gives ``closer''
inputs (since you only miss once.)
\end{enumerate}

\paragraph{Why test invalid inputs?} Bill Sempf (@sempf), on Twitter: ``QA Engineer walks into a bar. Orders a beer. Orders 0 beers. Orders 999999999 beers. Orders a lizard. Orders -1 beers. Orders a sfdeljknesv.''

If you're lucky, your program accepts strings (or events) described by a regexp or a grammar. But you might not use a parser or regexp engine. Generating using the regexp or grammar helps detect deviations, both right now and in the future.

What we'll do is to mutate the grammars and generate test strings from the mutated grammars.

Some notes:
\begin{itemize}
\item Can generate strings still in the grammar even after mutation.
\item Recall that we aren't talking about semantic checks.
\item Some programs accept only a subset of a specified larger language,
e.g. Blogger HTML comments. Then testing intersection is useful.
\end{itemize}

\end{document}
